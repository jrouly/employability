\begin{abstract}
  Entry level workforce positions requiring a minimum of postsecondary education is becoming more and more prevalent.~\cite{carnevale2010}
  Degrees are being used as a semi standard basis for knowledge and skill expectations of new employees.
  Consequently increasingly many young people are turning to postsecondary education as a means of becoming qualified for a job.
  In turn, educational institutions must ensure that curricula are suited to the expectations of educators, recruiters, and students.
  Is it possible to determine whether postsecondary institutions are preparing students for the workforce?
  This paper introduces an application of ``Natural Language Processing'' (NLP) to large corpora of human language documents in an effort to measure this.
  The contribution is a method of retrieving, cleaning, and processing these documents to then measure the degree of overlap between educational outcomes and workforce expectations.
\end{abstract}
