\section{Results}

\subsection{Topic Models}

The first result of performing LDA is the inferred topic models.
Because LDA is an unsupervised process there are no labels or summaries associated with the topics, only a nominal identifier.
To directly judge the semantics of a topic a human must interpret the terms and their associated frequency and extrapolate shared semantics.
Over several runs of LDA with various parameterization, different quality topics were extracted.
Table~\ref{tab:topics} displays a small selection of topics --- in this run, five hundred topics were inferred across the corpora.
The complete list of inferred topics is available online\footnote{\href{https://employability.rouly.net/web/topics}{employability.rouly.net/web/topics}}.

\begin{table}
  \centering
  \begin{tabular}{l r}
    \multicolumn{2}{c}{\small{\textbf{Topic 16}}} \\
    {\small\textit{Term}} & {\small\textit{Frequency}} \\
    \hline
    test & 0.0132 \\
    data & 0.0128 \\
    software & 0.0115 \\
    design & 0.0107
  \end{tabular}
  \begin{tabular}{l r}
    \multicolumn{2}{c}{\small{\textbf{Topic 273}}} \\
    {\small\textit{Term}} & {\small\textit{Frequency}} \\
    \hline
    market & 0.0250 \\
    product & 0.0113 \\
    strategi & 0.0107 \\
    digit & 0.0103
  \end{tabular}
  \begin{tabular}{l r}
    \multicolumn{2}{c}{\small{\textbf{Topic 418}}} \\
    {\small\textit{Term}} & {\small\textit{Frequency}} \\
    \hline
    branch & 0.0224 \\
    rental & 0.0167 \\
    enterpris & 0.0155 \\
    manageri & 0.0146
  \end{tabular}
  \begin{tabular}{l r}
    \multicolumn{2}{c}{\small{\textbf{Topic 424}}} \\
    {\small\textit{Term}} & {\small\textit{Frequency}} \\
    \hline
    care & 0.0346 \\
    nurs & 0.0179 \\
    health & 0.0168 \\
    clinic & 0.0117
  \end{tabular}
  \caption{Four inferred topics.}~\label{tab:topics}
\end{table}

\subsection{Domain Overlap}

From an LDA run with five hundred topics, only four topics were identified as ``strictly overlapping'' for a certain threshold.
Overlapping is used to mean the topic was present for a proportion of documents in either corpus at a rate greater than some threshold.
Strict overlapping is used similarly, but the topic must be present in both corpora at a rate greater than the threshold.
Table~\ref{tab:overlap} displays the strictly overlapping topics for the highest possible overlap threshold.
Notice that the topics in Table~\ref{tab:topics} and Table~\ref{tab:overlap} are the same.

The Jaccard index of these two sets can be found by dividing the cardinality of the strictly overlapping set (ostensibly the intersection) by the number of extracted topics.
This gives us an index of 0.044, which is an exceptionally low measure of the similarity between the career topics and curricular topics.

\begin{table}
  \centering
  \begin{tabular}{l r r r r}
    & \multicolumn{4}{c}{\small{\textbf{Topic Expression}}} \\
    & \multicolumn{2}{c}{\small{\textbf{Career Data}}} & \multicolumn{2}{c}{\small{\textbf{Curricular Data}}} \\
    {\small\textit{Topic}} & {\small\textit{Count}} & {\small\textit{\%}} & {\small\textit{Count}} & {\small\textit{\%}} \\
    \hline
    16 & 9,571 & 2.57 & 620 & 1.31 \\
    73 & 9,377 & 2.52 & 648 & 1.37 \\
    18 & 4,713 & 1.27 & 580 & 1.23 \\
    24 & 6,181 & 1.66 & 568 & 1.20
  \end{tabular}
  \caption{Strictly overlapping topics for a threshold of 1\%.}~\label{tab:overlap}
\end{table}
